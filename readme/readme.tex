\documentclass[UTF8]{ctexart}
%\documentclass{article}
%\usepackage{CJK}
\usepackage {Hologo}
\usepackage {listings}
\usepackage{fontspec}  
\setmonofont{Consolas} 
\begin{document}
\title{同济大学本科毕业论文\Hologo{LaTeX}模板说明}
\author{\textbf{HarlanHW,harlanhw@foxmail.com}}
%\date{today}
\maketitle
\subsection{说明}笔者为2019届同济本科毕业生,根据之前学长险些失传的同济大学本科毕业论文\Hologo{LaTeX}模板调整,改进,补充为2019年毕业论文模板,并对原项目中的一些问题给予修正。
\subsection{编译环境}
笔者使用的编译环境为 Windows10+texlive2017+CTeX 2.9.2
\subsection{问题和解决方案}
下面是遇到的一些问题。
\subsubsection{缺少ccmap.sty}
这个问题会报错但是不影响编译结果,可以自己新建一个ccmap.sty放在对应位置。
\subsubsection{缺少slashbox.sty}
从http://mirrors.cqu.edu.cn/CTAN/macros/latex/contrib/slashbox/slashbox.sty下载slashbox.sty放在对应的位置。
\subsubsection{缺少gbk2uni.exe}
由于原文件用的是gbk编码,在转成PDF时容易乱码需要用到gbk2uni.exe,已经补充在项目中。使用时需要将其复制到C:$\backslash$*$\backslash$CTEX$\backslash$MiKTeX$\backslash$miktex$\backslash$bin中,其中*为自己对应的路径。
\subsubsection{可能出现的一个问题}
使用WinEdt要注意在Options->Execution Modes->Tex System里设置Tex Bin的路径为你安装的路径,格式为C:$\backslash$*$\backslash$CTEX$\backslash$MiKTeX$\backslash$miktex$\backslash$bin
\subsection{致谢}
感谢该模板最初的来源清华大学T\small{HU}T\small{HESIS}开发者薛瑞尼以及同济大学贡献人员:\textbf{闻人Q},\textbf{WildWolfxg},\textbf{Gondalman},\textbf{j0sf}。

\end{document}

